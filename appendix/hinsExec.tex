\section{More about High-level Instructions Execution}
\label{appendix:more-about-high-level-insExec}

We give some supplements about the execution of high-level instructions. 
As we have explained in \Sec{\ref{subsec:High-level Pseudo-SPARCv8 Language}}, 
the register windows and delayed buffer in pyhsical SPARCv8 program state 
are omitted in high-level Pseudo-SPARCv8 program state. So, we do not define  
state transtion rules for instructions $\csave{}$, $\crestore{}$, $\rd{}$, 
and $\cwr{}$. The instruction transition rules for the rest of instructions, like  
$\ld{}$ and $\cadd{}$, have no much difference with the rules in pyhical SPARCv8 program. 
\begin{figure}[!h]
    \centering
    \[
        \begin{array}{c}
            \infer
            {
                \hexeci{\ld \ \aexp \ \reg{d}}{((\hRfile, \hWstk), \Mem)}
                    {((\hRfile', \hWstk), \Mem)}
            }
            {
                \evalR{\aexp}{\hRfile} = \loc \quad \ \ 
                \Mem(\loc) = \val \quad \ \ 
                \hRfile' = \hRfile\hRegUpd{ \reg{d} \rightsquigarrow \val } 
            } \\
            \\[-5pt]
            \infer
            {
                \hexeci{(\cadd \ \reg{s}, \oexp, \reg{d}}{((\hRfile, \hWstk), \Mem))}
                    {((\hRfile', \hWstk), \Mem)}
            }
            {
                \hRfile(\reg{s}) = \val_1 \quad \ \ 
                \evalR{\oexp}{\hRfile} = \val_2 \quad \ \ 
                \reg{d} = \dom(\hRfile) \quad \ \ 
                \hRfile' = \hRfile\hRegUpd{ \reg{d} \rightsquigarrow \val }
            }
        \end{array}
    \]
    \caption{Transition rules for instructions \ld{} and \cadd{} in high-level}
    \label{fig:Transition rules for instructions ld and add in high-level}
\end{figure}

We show the state transition rules for instructions \ld{} and \cadd{} 
in high-level in \Fig{\ref{fig:Transition rules for instructions ld and add in high-level}}. 
The register file updating operation is defined formally below : 
\[
    \hRfile\hRegUpd{\hrn \rightsquigarrow \val} \ \define \ 
    \hRfile\{ \hrn \rightsquigarrow \val \} \quad \ \ 
    \text{where } \, 
    \hrn \notin \{ \spreg, \fpreg \}
\] 
According to the definition, we can find that updating the register $\spreg$ 
(alias of $\reg{14}$), which is used to point to the top of the current 
stack frame, and $\fpreg$ (alias of $\reg{14}$), which is used to point to 
the top of the previous stack frame is not allowed. Only the execution of instructions  
$\Psave$, which is used to allocate a new stack frame, 
and $\Prestore{}$, which is used to free the current stack frame, can modify them. 
The evaluation of the opand and address expression in high-level is defined formally 
below : 
\[
    \begin{array}{ll}
        \begin{array}{lcl}
            \evalR{\oexp}{\hRfile} & \define &
                \left\{
                    \begin{array}{ll}
                        R(r) &\quad \cif \ \oexp = r \\
                        \\[-8pt]
                        w &\quad \cif \ \oexp = \word, \\
                        & \quad \quad -4096 \leq \word \leq 4095 \\
                        \\[-8pt]
                        \perp &\quad \otherwise
                    \end{array}
                \right.
        \end{array} & \quad
        \begin{array}{lcl}
            \evalR{\aexp}{\hRfile} & \define &
                \left\{
                    \begin{array}{ll}
                        \evalR{\oexp}{\hRfile} &\quad \cif \ \aexp = \oexp \\
                        \\[-8pt]

                        \val_1 \!+\! \val_2
                        &\quad \cif \ \aexp = \regr \!+\! \oexp,\
                        \hRfile(\regr) \!=\! \val_1 \\
                        & \quad \quad \tand \ \evalR{\oexp}{\hRfile} = \val_2 \\

                        \\[-8pt]
                        \perp &\quad \otherwise
                    \end{array}
                \right.
        \end{array}
    \end{array}
\]

The `` $\Psave \ \word$ " can be viewed as a macro of 
`` $\csave{} \ \spreg, -\word, \spreg$ ", and `` $\Prestore{}$ " 
can be viewed as a marco of `` $\crestore{} \ \greg{0}, \greg{0}, \greg{0}$ ". 
\begin{figure}[h]
    \centering
    \[
        \begin{array}{c|c}
            \begin{array}{lcl}
                \csave{} & \quad & \spreg, -128, \spreg \\
                \cadd{} & & \ireg{0}, \ireg{1}, \ireg{0} \\
                \ret{} \\
                \crestore{} & & \greg{0}, \greg{0}, \greg{0} \\
                & (a)
            \end{array} \qquad & \qquad 
            \begin{array}{lcl}
                \Psave & \quad & 128 \\
                \cadd{} & & \ireg{0}, \ireg{1}, \ireg{0} \\
                \ret{} \\
                \Prestore{} & & \\
                & (b)
            \end{array}
        \end{array}
    \]
    \caption{Realistic SPARCv8 Code and Pseudo-SPARCv8 Code}
    \label{fig:Realistic SPARCv8 Code and Pseudo-SPARCv8 Code}
\end{figure}

\Fig{\ref{fig:Realistic SPARCv8 Code and Pseudo-SPARCv8 Code}} 
gives a simple comparision with the realistic SPARCv8 code 
and our Pseudo SPARCv8 code in high-level. 
\Fig{\ref{fig:Realistic SPARCv8 Code and Pseudo-SPARCv8 Code}} 
(a) is the realistic SPARCv8 code. 
It uses instruction `` $\csave{} \, \spreg, -128, \spreg$ " 
to store the caller's context and allocate a new stack frame 
size 128 bytes for the current procedure, and use instruction
`` $\crestore{} \ \greg{0}, \greg{0}, \greg{0}$ " to restore 
the caller's context at the exitance of the current procedure. 
\Fig{\ref{fig:Realistic SPARCv8 Code and Pseudo-SPARCv8 Code}} (b) is 
the same function in Pseudo-SPARCv8 code, 
and we can find that the instructions that 
is responsible for saving and restoring the context of caller
is replaced by `` $\Psave \ 128$ " and `` $\Prestore$ ". 