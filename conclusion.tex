\section{Related Work and Conclusion}
\label{sec:conclusion}

There has been much work on assembly or
machine code verification. Most of them
do not support function calls or simply
treat function calls in the continuation-passing
style where return addresses are viewed as first
class code pointers~\cite{PCC,FPCC,TAL,TALx86,Yu03ESOP,xcap,cflogic}.
SCAP~\cite{Feng06pldi} supports assembly code verification
with various stack-based control abstractions, including
function call and return. We follow the same idea here.
However, SCAP gives a syntactic-based soundness proof
by establishing the preservation of the syntactic judgment,
which makes it difficult to interact with other modules
verified in different logic. Since our goal is to
verify inline assembly and link the verified code
with the verified C programs, we give a direct-style
semantic model of the logic judgments. And it allows us 
to extend our program logic to support verifying 
contextual refinement without meet much challenges. 
Also SCAP
is based on a simplified subset of assembly instructions,
while our work is focused on a realistically modeled
subset of \sparc{} instructions.

In terms of the support of realistic instruction sets,
previous work on proof-carrying code (PCC) and
typed assembly language (TAL) mostly supports subsets of
x86.
Myreen's work \cite{arm-veri} presents a framework for
ARM verification based on a realistic model
(but it doesn't support function call and return).

As part of the Foundational Proof-Carrying Code (FPCC)
project~\cite{FPCC},
Tan and Appel present a program logic $\mathcal{L}_c$
for reasoning about control flow in assembly code~\cite{cflogic}.
Although $\mathcal{L}_c$ is implemented on top of SPARC machine
language, the underlying logic is a type system instead
of a full-blown program logic for functional correctness.
It reasons about functions in the continuation-passing
style. Also 
handling SPARC features such as delayed writes or delayed
control transfers is not the focus of $\mathcal{L}_c$.
%They also reason about functions in the continuation-passing
%style.
There has been work on mechanized semantics of the SPARCv8 ISA.
%Here are already some works for SPARCv8 ISA formalization.
\etal{Hou}~\cite{sparcv8-formalization-Isabelle} model the SPARCv8 ISA
in Isabelle/HOL, and test their formal model against 
LENON3 simulation board, 
which is a synthesisable VHDL model of a 32-bit processor 
compliant with the SPARCv8 architecture, 
through more than 100,000 instruction instances.   
\etal{Wang}~\cite{sparc-formalization} formalize its semantics in Coq.
%However, both of their works give no program logic.
Our operational semantics of SPARCv8 follows \etal{Wang}~\cite{sparc-formalization}.
%\etal{Wang}~\cite{sparc-formalization}
%formalize the SPARCv8 ISA in Coq but give
%no program logic. Our operational
%semantics of SPARCv8 follows their work. 
But \etal{Wang} do not validate their formalization against actual 
hardware, we remain it as a future work. 

\etal{Ni}~\cite{ctxm} verify a context switch module of 19 lines
in x86 code to show case the support of embedded
code pointers (ECP) in XCAP~\cite{xcap}. 
We use our program logic to verify the contextual 
refinement between a context switch routine in \sparc{} 
and $\primsw{}$ primitive. The context switch routine 
implemented in \sparc{} that we verified 
is more complicated then 
implemented in x86, because of the requirement to save 
the contexts stored in register window in memory.
% The context switch
% module we verify comes from a practical OS kernel,
% which is more realistic and consists of more than 250 lines
% of assembly code, but our logic (\textbf{CALL} rule) 
% does not really support the switch of return pointers,   
% which requires further extension like OCAP~\cite{FengTLDI07}. 
% Our focus is to verify the
% code manages the register windows, and the function calls made internally. 
% We address this problem of calling context switch routine in 
% another way in our work, such that we can use the extended 
% program logic to verify the contextual refinement between 
% context switch routine and $\primsw{}$ primitive. 
% The method of verifying context switch primitive in our work is 
% inspired by the approach used by \etal{Guo}~\cite{ctxmGuo}
% that protects the TCBs through abstraction, but the state 
% relation between low- and high-level program state in our work is more 
% sophisticated than theirs, because of some specific mechanisms, 
% like register windows, in SPARCv8. 
% They also do not develop a general program 
% logic for refinement verification of the assembly.  
% Our extended program logic is based on the relational program logic, 
% which has already been well used 
% successfully in the refinement verification 
% of C-style language \cite{Xu16cav,liang13pldi,liang14lics}.  

Yang and Hawblitzel~\cite{YangPLDI10} verify Verve, an x86 
implementation of an experimental operating system. Verve has two 
levels, the high-level TAL code and the low-level ``Nucleus'' 
that provides primitive access to hardware and memory. 
The Nucleus code is verified automatically using the Z3 SMT solver,
while the goal of our work is to generate machine checkable proofs.
Another key difference is the use of different ISAs. Here
we give details to verify specific features of SPARCv8 programs.

There have been many techniques and tools proposed for automated
program verification (\eg~\cite{SymbolicExecutionSpLogic,Smallfoot}).
It is possible to adapt them to verify SPARCv8 code.
We propose a new program logic and do the verification in Coq mainly
because the work is part of a big project for a fully certified OS 
kernel for aerospace crafts whose inline assembly is written in 
SPARCv8. We already have a program logic implemented in Coq for 
C programs, which allows us to verify C code with Coq proofs. 
Therefore we want to have a program logic for SPARCv8 so that 
it can be linked with the correctness proof of 
C and can generate machine-checkable Coq proofs too.
That said, many of the automated verification techniques can
be applied to reduce the manual efforts to write Coq proofs,
which we would like to study in the future work. 
% We have already shown that it's possible to apply some 
% Coq tactics based on separation logic \cite{practical-tactics} 
% in our work.  


% %Most of our code verification works are done by hand now.
% %\etal{Berdine}~\cite{SymbolicExecutionSpLogic}\cite{Smallfoot} describe a
% %method for proving Hoare triples by a form of symbolic execution,
% %in order to achieve automatical reasoning. Their symbolic execution algorithm
% %works by proof-search using operational and rearrangement rules.
% %%Our current work consider less about automatical reasoning,
% %%but it's possible to do much effort on it in the future.
% %The operational rules in Berdine's work are similar with
% %traditional Hoare logic rules with some certain restrictions on pre- and post-condition,
% %and their rearrangement rules can be viewed as some derived rules from the basic ones.
% %So, in order to use their algorithm to achieve automatic reasoning of SPARCv8 code,
% %we have to design a Hoare-style program logic for SPARCv8 code firstly,
% %which is the aim of our work.
% %%With a Hoare-style program logic for SPARCv8 code,
% %%their algorithm can be implemented as a Coq tactic.
% %Although, our current work consider less about automatical reasoning,
% %it's possible to extend it and their algorithm can be implemented as a Coq tactic
% %based on our Hoare-style program logic for SPARCv8 code in the future.
% %
% %\indent
% %There are many impressive pioneering works
% %that focus on machine code verification.
% %Projects on proof-carrying code (PCC) \cite{PCC, FPCC}
% %and typed assembly language (TAL) \cite{TAL, TALx86}
% %evoke interests in low-level code verification.
% %They aim to find a method to address safety properties
% %of assembly code by automatic type-checking.
% %Generating proofs automatically is difficult in many times.
% %So, the work \cite{Yu03ESOP} of \etal{Yu} provides the CAP framework
% %to support a Hoare-logic style reasoning for assembly code.
% %However, works on PCC, TAL and CAP only focus on
% %the safety properties of assembly code.
% %
% %\indent
% %Certifying function call/return in assembly code is difficult,
% %because assembly codes are usually lacking structure.
% %\etal{Feng} develop SCAP framework \cite{Feng06pldi}
% %for verification of assembly code with all kinds of
% %stack-based control abstraction,
% %including function call/return.
% %Our method to handle function call/return is inspired by SCAP.
% %%
% %However, most works on CAP, including SCAP,
% %are based on a simply abstract machine model
% %and establish soundness following the syntactic approach
% %of proving type sound.
% %And they also do not give the frame rule for local verification.
% %Myreen's work \cite{arm-veri} presents a framework for
% %ARM verification based on a realistic model
% %and can state the functional behavior of ARM program,
% %but doesn't support function call/return verification.
% %
% %work \cite{cflogic}
% %which is a part of Foundational Proof-Carrying Code (FPCC) \cite{FPCC}
% %presents a program logic about reasoning unstructured control flow
% %in machine-language program.
% %Although their logic is implemented
% %on the top of SPARC machine language and
% %provides fine-grained composition rules to compose program segment,
% %the main target of their work is not to design a program logic
% %specially for SPARC code.
% %They do not consider the main features of SPARC in their logic
% %and still treat function call and return as the first-class function.
% %
% %
% %\indent
% %Following the CAP framework,
% %\etal{Ni} present XCAP \cite{xcap} to solve
% %embedded code pointers (ECPs) problem in assembly code level,
% %and \etal{Feng} extend the basic CAP to handler
% %concurrent multi-threaded assembly code verification.
% %Our current work is not support ECPs
% %and concurrency verifications \cite{Feng05ICFP, Feng08pldi, Yu04ICFP}.
% %So, these work can be the complementaries in our future work.
% %
% %\indent
% %The existent works about SPARC are less.
% %\etal{Wang} \cite{sparc-formalization}
% %formalize the SPARCv8 ISA in Coq.
% %Their work does not develop a program logic
% %and uses the operational semantics for code reasoning directly.
% %Tan and Apple's work \cite{cflogic}
% %which is a part of Foundational Proof-Carrying Code (FPCC) \cite{FPCC}
% %presents a program logic about reasoning unstructured control flow
% %in machine-language program.
% %Although their logic is implemented
% %on the top of SPARC machine language and
% %provides fine-grained composition rules to compose program segment,
% %the main target of their work is not to design a program logic
% %specially for SPARC code.
% %They do not consider the main features of SPARC in their logic
% %and still treat function call and return as the first-class function.
% %
% %\indent
% %Context switch module is an important component in OS kernel.
% %There exists many works to verify a context switch module
% %that \etal{Ni} have used the XCAP framework
% %to certify a machine context management in x86 \cite{ctxm}.
% %However, the programs they proved are implemented by them
% %and context switch module is only 19 lines in their work.
% %We prove a realistic context switch module
% %which is more complex and considers more details
% %and cases for actual execution.
% %Although we does not consider ECPs problem
% %which XCAP can handler well,
% %it does not influence the context switch module that we verified
% %to link with other modules,
% %because it's a function and our logic supports
% %function call/return verification.

\paragraph{\textbf{Conclusion.}}
We present a relational program logic for \sparc.
Our logic is based on a realistic semantics
model and supports main features of \sparc,
including delayed control transfer, delayed writes,
and register windows.
It also supports modular reasoning of 
function call in a direct-style and 
refinement verification. 
We apply our logic to verify  
that there is a contextual refinement between 
a context switch routine implementated 
in \sparc{} and $\primsw$ primitive for task switching.
% And we also extend the program logic to support 
% refinement verification and apply the extended 
% program logic to verify 
% that there is a contextual refinement between 
% a context switch routine implementated 
% in \sparc{} and $\primsw$ primitive for task switching. 
% We have applied the program logic to verify
% the main body of the context switch routine
% in a realistic embedded OS kernel. 
% And we also extend the program logic to support 
% refinement verification. 
Our current work can only handle
sequential \sparc{} program verification and 
do not consider interrupt in machine model.  
We will extend it for concurrency verification 
and finish the step {\color{blue} \textcircled{1}} shown in
\Fig{\ref{fig:idea to establish contextual refinement}}
that the compilation can ensure the behaviors 
of the Pseudo-SPARCv8 code calling abstraction assembly 
primitives in intermediate level refines 
the behaviors of 
the client C code calling abstract assembly primitives 
in source level in the future. 
% %and accurate models of \sparc{} program,
% %and supports module, local, function call/return
% %and main features of \sparc{} verification.
% %We also give a semantic approach for soundness establishing
% %and prove the soundness of our logic.
% %We finally apply it to verify a context switch module
% %which has been used in an actual embedded OS kernel.
% % Our current work can only handle
% % sequential \sparc{} program verification for partial correctness.
% % We will extend it for concurrency
% % and refinement verification in the future. Also
% % we would like to link the verified inline assembly
% % with verified C code for whole system
% % verification.
