\section{\sparc{} Assembly Language}
\label{sec:modeling}
\begin{figure*}[!t]
	\centering
	\small
	\[
		\begin{array}{llllcllll}
			\TYPE{(Word)} & \word, \lab{} & \in &
			\multicolumn{6}{l}
			{
				\TYPE{Int32} \qquad
				\TYPE{(Block)} \ \block \, \in \, \Ztype
				\qquad
				\TYPE{(Addr)} \ \loc \, \in \,
					\TYPE{Block} \times \TYPE{Word}
				\qquad
				\TYPE{(Val)} \ \val \,
					\define \, \word \sepline \loc
			}
			\\
			\\
			\TYPE{(Prog)} & \prog & \define &
				(\code, \state, \pc, \npc) & \qquad &
			\TYPE{(CodeHeap)} \ \ & \code \ & \in &
				\TYPE{Word} \rightharpoonup \TYPE{Comm}
			\\
			\\[-9pt]
			\TYPE{(State)} & \state & \define &
				(\mem, \Rstate, \DBuf) & &
			\TYPE{(RState)} & \Rstate & \define &
				(\RFile, \Wstack)
			\\
			\\[-9pt]
			\TYPE{(Mem)} & \mem & \in &
				\TYPE{Addr} \rightharpoonup \TYPE{Val}
			& &
			\TYPE{(ProgCount)} & \pc, \npc & \in & \TYPE{Word}
			\\
			\\[-9pt]
			\TYPE{(OpExp)} & \oexp & \define &
				\reg{} \sepline \word & &
			\TYPE{(AddrExp)} & \aexp & \define &
				\oexp \sepline \reg{} + \oexp \\
			\\
			\TYPE{(Comm)} & \comm & \define &
			\multicolumn{6}{l}
			{
				\simplins{} \sepline \call{} \ \lab{}
				\sepline \jmp{} \ \aexp \sepline \retl{} \sepline
				\be \ \lab{}
			} \\
			\\[-9pt]
			\TYPE{(SimpIns)} \ & \simplins{} & \define &
			\multicolumn{6}{l}
			{
				\ld{} \ \aexp \ \reg{d} \sepline
				\st{} \ \reg{s} \ \aexp \sepline
				\nop{} \sepline
				\cadd{} \ \reg{s} \ \oexp \ \reg{d} \sepline
				\csave{} \ \reg{s} \ \oexp \ \reg{d} \sepline
				\crestore{} \ \reg{s} \ \oexp \ \reg{d}
			} \\
			& & \ \ \ \ | &
			\multicolumn{6}{l}
			{
				\rd{} \ \sr \ \reg{d} \sepline
				\cwr{} \ \reg{s} \ \oexp \ \sr \sepline
				\dots
			} \\
			\\[-9pt]
			\TYPE{(InstrSeq)} & \cblk & \define &
			\multicolumn{6}{l}
			{
				\simplins{}; \, \cblk \sepline
				\jmp{} \ \aexp; \, \simplins{} \sepline
				\call{} \ \lab{}; \simplins{}; \cblk \sepline
				\retl{} \ \simplins{} \sepline
				\be{} \ \lab{}; \, \simplins{}; \, \cblk
			}
		\end{array}
	\]
	\vspace*{-0.5em}
	\caption{Machine States and Language for SPARCv8 Code}
	\label {fig:Machine States and Language for SPARC Code}
	\vspace*{-0.5em}
\end{figure*}

We introduce the key \sparc{} instructions, the model of machine
states, and the operational semantics in this section.

\subsection{Language Syntax and States}
\label{subsec:syntax}

The machine model and syntax of \sparc{} assembly language
are defined in Fig.~\ref{fig:Machine States and Language for SPARC Code}.
Here, we follow the block-based memory \cite{CompCertMM} introduced
in CompCert to define the memory in our work.
% so that each procedure can have a block of its own
% stack frame.
The memory address $\loc$ is defined as a pair of
its block id and the offset. Block ids
(\TYPE{Block}) are integers
in mathemantics presented as $\Ztype$, 
and offsets (\TYPE{Word})
are 32-bit integers (called ``words").
The value (\TYPE{Val})
is either a word $\word$ or address $\loc$.
The whole program configuration (\TYPE{Prog})
$\prog$ consists of the code heap (\TYPE{CodeHeap})
$\code$, the machine state (\TYPE{State}) $\state$, 
and the program counters $\pc$ and $\npc$.
%\sparc{} program can be viewed as a composition of
%code heap, machine state, pc and npc.
The code heap $\code$ is a partial function
from labels $\lab{}$ to commands $c$.
Labels are also 32-bit integers (called ``words"),
which can be viewed as addresses or locations
where the commands are saved in the code heap.
The operand expression (\TYPE{OpExp})
$\oexp$, which is either a general
register $\reg{}$ or a word $\word$,
and address expression (\TYPE{AddrExp}) $\aexp$,
which is either an operand expression or a
sum of the values of register $\reg{}$ and an operation
expression, are auxiliary definitions used as parameters of commands.
Commands (\TYPE{Comm}) in \sparc{} can be classified into two categories:
(1) simple instructions (\TYPE{SimpIns})
$\simplins{}$, which do sequential
operation, \eg{} arithmetic operation ``\cadd{}",
or memory operations
``\ld{}" (load) and ``\st{}" (store), or register window
operations ``\csave{}" and ``\crestore{}", or special
register operations ``\rd{}" (read) and ``\cwr{}" (write), 
or ``\nop{}", whose execution changes no program state
% (except assigning \npc{} to \pc{} and (\npc{}+4) to \npc{});
(the program counters \pc{} and \npc{});
(2) control-transfer instructions,
\eg{} \call{} and \retl{} for
function call and return, or \jmp{} and \be{} for
unconditional and conditional branch.

The machine state (\TYPE{State}) $\state$ consists of three parts:
the memory (\TYPE{Mem}) $\mem$, 
the register state (\TYPE{Rstate}) $\Rstate$
which is a pair of register file 
(\TYPE{RegFile}) $\RFile$ and frame list (\TYPE{FrameList}) $\Wstack$,
and the delay buffer (\TYPE{DelayBuff}) $\DBuf$.
As defined in Fig.~\ref{fig:Register File and Frame List},
$\RFile$ is a partial mapping
from register names (\TYPE{RegName}) to values.
Registers include the general registers $\regr$,
the processor state registers (\TYPE{PsrReg}) $\psr$
and the special registers (\TYPE{SpeReg}) $\sr$.
$\psr$ contains
the integer condition code fields $\regn$,
$\regz$, $\regv$ and $\regc$,
which can be modified by the arithmetic and logical instructions
and used for conditional control-transfer,
and $\regcwp$ recording the id of the current register window.
We explain the frame list $\Wstack$ and the delay buffer
$\DBuf$ below.

\begin{figure*}[!t]
	\small
	\[
		\begin{array}{llllcllll}
			\TYPE{(RegFile)} & \RFile & \in & 
			\TYPE{RegName} \rightharpoonup \TYPE{Val}
         & \quad \ \ &
        \TYPE{(RegName)} & \regN & \define &
		 \reg{0} \sepline \dots \sepline \reg{31} \sepline \psr \sepline \sr \\
		 \\[-9pt]
		
		\TYPE{(PsrReg)} & \psr & \define &
		\regn \sepline
			\regz \sepline
			\regv \sepline
			\regc \sepline \regcwp & \quad &
        %		
		\TYPE{(SpeReg)} & \sr & \define &
		\regwim \sepline
			\regY \sepline
\asr_0 \sepline \dots \sepline \asr_{31} \\ \\[-9pt]
		
		\TYPE{(FrameList)} \  & \Wstack & \define &
            \nil \sepline \fm \stCons \Wstack & \quad &
		\TYPE{(Frame)} & \fm & \define & [\val_0, \dots, \val_7] 
		\\ \\[-9pt]
		
		\TYPE{(DelayBuff)} & \DBuf & \define & \nil \sepline
                                  (\tick, \sr, \word) \dbCons \DBuf
                                  & \quad
                                  %& & & & \\		
		& \TYPE{(DelayCycle)} \ & \tick & \in & \{ 0, 1, \dots, X \}
		\end{array}
	\]
% 	\small
% 	\begin{tabular}{rcclcrccl}
% 		(RegFile) & $\RFile$ & $\in$ & $\text{RegName} \rightharpoonup
% 			\text{Val}$

%          & $\quad$ &
%         (RegName) & $\regN$ & $\define$ &
%          $\reg{0} \sepline \dots \sepline \reg{31} \sepline \psr \sepline \sr$ \\
		
% 		%(GenReg) & $r$ & $\define$ &
% 		%$\reg{0} \sepline \dots \sepline \reg{31}$ & $\quad\quad$ &
% 		(PsrReg) & $\psr$ & $\define$ &
% 		$\regn \sepline
% 			\regz \sepline
% 			\regv \sepline
% 			\regc \sepline \regcwp$ & $\quad$ &
%         %		
% 		(SpeReg) & $\sr$ & $\define$ &
% 		$\regwim \sepline
% 			\regY \sepline
% \asr_0 \sepline \dots \sepline \asr_{31}$ \\
		
% 		(FrameList) & $\Wstack$ & $\define$ &
%                             $\nil \sepline \fm \stCons \Wstack$ & $\quad$ &
% 		(Frame) & $\fm$ & := & $[\val_0, \dots, \val_7]$ \\
		
% 		(DelayBuff) & $\DBuf$ & $\define$ & $\nil \sepline
%                                   (\tick, \sr, \word) \dbCons \DBuf$
%                                   & $\quad$
%                                   %& & & & \\		
% 		%(DelayItem) & $d$ & $\define$ & $(\tick, \sr, \word)$ & $\quad$
% 		& (DelayCycle) & $\tick$ & $\in$ & $\{ 0, 1, \dots, X \}$
% 	\end{tabular}
	\caption{Register File, Frame List and DelayBuffer}
	\label{fig:Register File and Frame List}
\end{figure*}

\paragraph{\textbf{Register windows and frame List.}}
\sparc{} provides 32 general registers that are split into
four groups
as \globalRN{} ($\reg{0} \sim \reg{7}$),
\outRN{} ($\reg{8} \sim \reg{15}$), \localRN ($\reg{16} \sim \reg{23}$)
and \inRN{} ($\reg{24} \sim \reg{31}$) registers.
The latter three groups (\outRN{}, \localRN{} and \inRN{})
form the current register window.

At the entry and exit of functions and traps, one may need to
save and restore some of the general registers as execution
contexts. Instead of saving them into stacks in memory,
\sparc{} uses multiple register windows to form a circular
stack, and does window rotation for efficient context save and restore.
As shown in Fig.~\ref{fig:RegisterWindows}
(the figure taken from ``The SPARC Architecture Manual Version 8".\footnote{The SPARC Architecture Manual Version 8. 
\url{https://gaisler.com/doc/sparcv8.pdf}.}), 
there are $N$
register windows ($N=8$ here) consisting of $2\times N$
groups of registers (each group containing 8 registers).
The $\regcwp$ register (part of $\psr$) records the id number
of the current window ($\regcwp=0$ in this example).
\begin{figure*}[!t]
	\centering
	\includegraphics[width=8.7cm]{window}
	% \includegraphics[width=58ex]{window}
	\caption{Register Windows}
	\label{fig:RegisterWindows}
\end{figure*}

The \inRN{} and \outRN{} registers of each window are shared
with its adjacent windows for parameter passing.
For example, the \inRN{} registers of $\text{w}_0$
are the \outRN{} registers of $\text{w}_1$,
and the \outRN{} registers of $\text{w}_0$
are the \inRN{} registers of $\text{w}_7$.
This explains
why we need only $2\times N$ groups of registers for
$N$ windows, while each window consists of
three groups (\outRN{}, \localRN{} and \inRN{}).

To save the context, the $\csave$ instruction
rotates the window by decrementing the $\regcwp$ pointer
(modulo $N$).
So $\text{w}_7$ becomes the current window. The \outRN{}
registers of $\text{w}_0$ become
the \inRN{} registers of $\text{w}_7$.
The \inRN{} and \localRN{} registers of $\text{w}_0$
become inaccessible.
This is like pushing them
onto the circular stack.
The $\crestore$ instruction does the inverse, which is like
a stack pop.

The $\regwim$ register is used as a bit vector to record the
end of the stack. Each bit in $\regwim$ corresponds to a
register window. The bit corresponding to the last available
window is set to 1, which means ``invalid". All other bits
are 0 (i.e., ``valid").
%Given a window pointer $\cwp$ in $\regcwp$,
When executing $\csave$ (and $\crestore$), we need to ensure
the next window is valid, in order to avoid the overflow
of register window because of the limitation of the number
of windows. We use the assertion $\winvalid(\cwp, \RFile)$ defined
in Fig.~\ref{fig:save and restore} to say the
window pointed to by $\cwp$ is valid, given the value of
$\regwim$ in $\RFile$.
%$$
%\winvalid(\cwp, \RFile) \ \define \ 2^{\cwp}  \& \RFile(\regwim) = 0
%$$
%where $\&$ represents bitwise AND.

We use the frame list $\Wstack$ to model the circular stack consisting
of register windows. As defined in Fig.~\ref{fig:Register File and Frame List},
a frame is an array of 8 words, modeling a group of 8 registers.
$\Wstack$ consists of a sequence of frames
corresponding to all the
register windows except the \outRN{}, \localRN{} and \inRN{}
registers in the current window. Then $\csave$ saves the
\localRN{} and \inRN{} registers onto the head of $\Wstack$
and loads the two groups of registers at the ``tail" of $\Wstack$
to the \localRN{} and \outRN{} registers (and the original
\outRN{} registers become the \inRN{} group). The $\crestore$
instruction does the inverse. The operations are defined formally
in Fig.~\ref{fig:save and restore}.
Here, we use `` $\stCons$ " for
adding an element at the head of a list, and use `` $\lstApp{}$ "
for appending an element at the tail of a list.
\begin{figure*}[!t]
    \[
     \begin{array}{l}
     \outRN \ \define \  [\reg{8}, \dots, \reg{15}]
     \qquad
     \localRN \ \define \  [\reg{16}, \dots, \reg{23}] %
      \qquad
     \inRN \ \define \  [\reg{24}, \dots, \reg{31}]
     \\
     \\[-8pt]
     \RFile([\reg{i}, \dots, \reg{i+k}]) \ \define\
     [\RFile(\reg{i}), \dots, \RFile(\reg{i+k})]
     \\
     \\[-8pt]
     %
     \Rupd{\reg{i}, \dots, \reg{i+7}}{\fm} \ \define \
			R\{ \reg{i} \rightsquigarrow \val_0 \}
				\dots \{ \reg{i+7} \rightsquigarrow \val_7 \}
			\qquad
			\text{where }\ \ \fm = [\val_0, \dots, \val_7]
			% \\
            % \hspace*{28ex}\text{where }\ \ \fm = [\val_0, \dots, \val_7]
     \\
     \\ %[-8pt]
     %
     %\qquad \Rlocal \ \define\ [\RFile(\reg{8}), \dots, \RFile(\reg{15})] \\

        \begin{array}{lcl}
            \winvalid(\cwp, \RFile) & \define & 2^{\cwp}
											\,\bitAND\, \RFile(\regwim) = 0								
			\\
             & & \ \ \ \mbox{where $\bitAND$ is the bitwise AND operation.}
            \\
            \\[-5pt]
		    \precwp(\cwp) & \define & (\cwp + N - 1) \modOP N
            %\\
		    %\\[-8pt]
            \qquad\quad
		    \postcwp(\cwp) \ \define \ (\cwp + 1) \modOP N
        \end{array}
            \\
		    \\ %[-8pt]
            %
        \begin{array}{lcl}
            \decwin(\RFile, \Wstack) & \define &
            \left\{
            \begin{array}{ll}
                (\RFile', \Wstack')
                & \quad \text{if }
                             \cwp' = \precwp(\RFile(\regcwp)),
                             \winvalid(\cwp', \RFile), \\
                & \quad \ \ \
                             \Wstack = \Wstack'' \lstApp \fm_1 \lstApp \fm_2, \
                              \Wstack' = \RFile(\localRN)
                                          \stCons\RFile(\inRN)\stCons\Wstack'',\\
                & \quad \ \ \ \RFile'' =
                \RFile\{\inRN\rightsquigarrow \RFile(\outRN),
                        \localRN\rightsquigarrow \fm_2,
                        \outRN \rightsquigarrow \fm_1\},
                \\
                 & \quad \ \ \
                            \RFile' = \fsubst{\RFile''}\regcwp{\cwp'},
                 \\
                 %
                \perp &\quad \text{if }
                                  \lnot\winvalid(\precwp(R(\regcwp)), \RFile)
            \end{array}
            \right. \\
%            & & \where \ \ Q = (R, F)
            \\[-8pt]

            \incwin(\RFile, \Wstack) & \define &
            \left\{
            \begin{array}{ll}
                (\RFile', \Wstack')
                & \quad \text{if }
                             \cwp' = \postcwp(\RFile(\regcwp)),
                             \winvalid(\cwp', \RFile), \\
                & \quad \ \ \
                             \Wstack = \fm_1 \stCons \fm_2 \stCons \Wstack''  , \
                              \Wstack' = \Wstack''\lstApp \RFile(\outRN)
                                                  \lstApp \RFile(\localRN), \\
                                          %\stCons\RFile(\inRN)\stCons\Wstack''\\
                & \quad \ \ \ \RFile'' =
                \RFile\{\inRN\rightsquigarrow \fm_2,
                        \localRN\rightsquigarrow \fm_1,
                        \outRN \rightsquigarrow \RFile(\inRN)\},
                \\
                 & \quad \ \ \
                            \RFile' = \fsubst{\RFile''}\regcwp{\cwp'},
                 \\
                 %
                \perp &\quad \text{if }
                                  \lnot\winvalid(\postcwp(R(\regcwp)), \RFile)
            \end{array}
            \right.
        \end{array}
    \end{array}
    \]
    \caption{Auxiliary Definitions for Instruction \csave{} and \crestore}
    \label{fig:save and restore}
\end{figure*}

\paragraph{\textbf{The delay buffer.}}
The delay buffer $\DBuf$ is a sequence of delayed writes.
Because the  $\cwr$ instruction
does not update the target register immediately,
we put the write operation onto the delay buffer.
A delayed write is recorded as a triple consisting of
the remaining cycles $\tick$ to be delayed,
the target special register $\sr$ and the value
$\word$ to be written. 
Note that the value of a special register is restricted to 
only words, since the special registers are used to record the
state of processors, and it is impossible to
store memory addresses in them.
% Note that we restrict that
% the value of a special register can only be a word,
% because the special registers are used to record the
% state of processor, and it is impossible to
% store memory addresses in them.

\paragraph{\textbf{Instruction sequences.}}
%\indent \indent
We use an instruction sequence $\cblk$ to model
a basic block, \ie{} a sequence of commands ending
with a control transfer.
As defined in Fig.~\ref{fig:Machine States and Language for SPARC Code},
we require that a delayed control-transfer instruction
must be followed by  a simple instruction $\simplins$,
because the actual control-transfer occurs after
the execution of $\simplins$.
The end of each instruction sequence can only be
$\jmp$ or $\retl$  followed by a simple
instruction $\simplins$.
Note that we do not view the $\call$ instruction
as the end of a basic block, since the callee is
expected to return, following our direct-style
semantics for function calls.
We define $\code[\lab{}]$ to extract
an instruction sequence starting
from $\lab{}$ in $\code$ below.
\[
	\small
	C[\lab{}] =
	\left\{
		\begin{array}{ll}
			\simplins; \cblk &
			\quad \ \ \code(\lab{}) = \simplins \;
				\text{and} \; \code[\lab{}+4] = \cblk \\
			
			\\[-8pt]
			
			\comm; \simplins &
			\quad \ \
				\comm = \code(\lab{}) \; \text{and} \;
				\comm = \jmp \; \aexp \; \text{or} \; \retl
			\\ & \quad \ \ \text{and} \; C(\lab{}+4) = \simplins \\
			
			\\[-8pt]
			
			\comm; \simplins; \cblk &
			\quad \ \ \comm = \code(\lab{}) \;
			\text{and} \; c = \call \; \lab{} \; \text{or} \; \be \; \lab{} \\
			& \quad \ \ \text{and} \;
				\code(\lab{}+4) = \simplins \; \text{and} \;
				\code[\lab{}+8] = \cblk \\
			
			\\[-8pt]
			
			\perp & \quad \ \ \text{otherwise}
		\end{array}
	\right.
\]

\subsection{Operational Semantics}
\label{subsec : Operational Semantics}

The operational semantics is taken from
\etal{Wang}~\cite{sparc-formalization},
but we use a block-based memory model and
omit features like interrupts and traps.
We show the selected rules in Fig.~\ref{Selected Operational Semantics}.
\begin{figure*}
	\centering
	\subfigure[]
	{
		\begin{minipage}[b]{1\textwidth}
		% \small
		\scriptsize
		\[
			\infer
			{
				\LGlobTrans
					{(\code, (\mem, (\RFile, \Wstack), \DBuf), \pc, \npc)}
					{ \ \ }
					{(\code, (\mem', (\RFile'', \Wstack'), \DBuf''), \pc', \npc')}
				% \ptrans{((M, (R, F), D), \pc, \npc)}
				% 	{((M', (R'', F'), D''), \pc', \npc')}
			}
			{
				\begin{array}{l}
					\Dstep{(\RFile, \DBuf)}{(\RFile', \DBuf')} \\
					\cttrans{((\Mem, (\RFile', \Wstack), \DBuf'), \pc, \npc)}
						{((\Mem', (\RFile'', \Wstack'), \DBuf''), \pc', \npc')}
				\end{array}
			}
		\]
		\end{minipage}
	}
	
	\subfigure[]
	{	
		\begin{minipage}[b]{1\textwidth}
		% \small
		\scriptsize
        \[
            \infer
            {
                \cttrans{((\Mem, (\RFile, \Wstack), \DBuf), \pc, \npc)}
                    {((\Mem', (\RFile', \Wstack'), \DBuf'), \npc, \npc+4)}
            }
            {
                \code(\pc) = \simplins \quad \quad
					\swtrans{(\Mem, (\RFile, \Wstack), \DBuf)}{\; \; \simplins \; \;}
						{(\Mem', (\RFile', \Wstack'), \DBuf')}
            }
        \]

		\[
			\infer
			{
				\cttrans{((\Mem, (\RFile, \Wstack), \DBuf), \pc, \npc)}
					{((\Mem, (\RFile, \Wstack), \DBuf), \npc, \lab{})}
			}
			{
				\code(\pc) = \jmp \; \aexp \quad \; \;
				\eval{\aexp} = \lab{}
			}
		\]
		
		\[
			\infer
			{
				\cttrans{((M, (R, F), D), \pc, \npc)}
				{
					((M, (R\{\reg{15} \rightsquigarrow \pc\}, F), D), \text{npc}, \lab{})
				}
			}
			{
				C(\pc) = \call \; \lab{} \quad \; \;
					\reg{15} \in \text{dom}(R)
			}			
		\]
		
		\[
			\infer
			{
				\cttrans{((M, (R, F), D), \pc, \npc)}
				{
					((M, (R, F), D), \npc, \lab{}\!+\!8)
				}
			}
			{
				C(\pc) = \retl \quad R(\reg{15}) = \lab{}
			}
		\]

		\[\!\!\!\!\!\!\!
			\infer
			{
				\cttrans{((\mem, (\RFile, \Wstack), \DBuf), \pc, \npc)}
					{((\mem, (\RFile, \Wstack), \DBuf), \npc, \lab{})}
			}
			{
				\code(\pc) = \be \ \lab{} \quad
				\RFile(\regz) \neq 0 \quad      
				%\RFile(\regz) = \word \quad
				%\word \neq 0
			} \qquad
			\infer
			{\cttrans{((\mem, (\RFile, \Wstack), \DBuf), \pc, \npc)}
				{((\mem, (\RFile, \Wstack), \DBuf), \npc, \npc\!+\!4)}}
			{
				\code(\pc) = \be \ \lab{} \quad
				\RFile(\regz) = 0
			}
		\]
		\end{minipage}
	}
	
	\subfigure[]
	{
		\begin{minipage}[b]{1\textwidth}
			% \small
			\scriptsize
			\begin{minipage}{0.48\textwidth}
			\[
				\infer
				{
					\swtrans{(M, (R, F), D)}{\; \; \simplins \; \;}
						{(M', (R', F), D)}
				}
				{
					\itrans{(M, R)}{\simplins}{(M', R')}
				}
			\]
			\end{minipage}
			\begin{minipage}{0.48\textwidth}
				\[
					\infer
					{
						\swtrans{(M, (R, F), D)}
						{\cwr \; \reg{s} \; \oexp \; \sr}{(M, (R, F), D')}
					}
					{
						\begin{array}{l}
							\RFile(\reg{s}) = \word_1 \quad \; \;
							{\llbracket \oexp \rrbracket}_R = \word_2 \quad \; \;
							\word = \word_1 \!\xorOP\! \word_2 \\
							\sr \in \text{dom}(\RFile) \quad \; \; \DBuf' = \setdelay(\sr, w, D)
						\end{array}
					}
				\]
			\end{minipage}

            \centering
            \vspace{0.3cm}
            \begin{minipage}{1\textwidth}
                \infer
                {
                    \swtrans{(\Mem, (\RFile, \Wstack), \DBuf)}
                    {\csave \; \reg{s} \; \oexp \; \reg{d}}
                    {(\Mem, (\RFile\{ \reg{d} \rightsquigarrow \val' \}, \Wstack'), \DBuf)}
                }
                {
					\decwin{(\RFile, \Wstack)} =
						(\RFile', \Wstack') \quad
                    \eval{\oexp} = \val
					\quad
					\val' = \RFile(\reg{s})\!+\!\val
                    % \RFile'' =
                    %   \fsubst{\RFile'}{\reg{d}}{\RFile(\reg{s}) \!+\! \val}
                }
                \vspace{0.3cm}
                \infer
                {
                    \swtrans{(\Mem, (\RFile, \Wstack), \DBuf)}
                    {\crestore \; \reg{s} \; \oexp \; \reg{d}}
                    {(\Mem, (\RFile\{ \reg{s} \rightsquigarrow \val' \}, \Wstack'), \DBuf)}
                }
                {
                    \incwin{(\RFile, \Wstack)} = (\RFile', \Wstack') \quad
                    \eval{\oexp} = \val
					\quad
					\val' = \RFile(\reg{s})\!+\!\val
                    % \RFile'' =
                    %   \fsubst{\RFile'}{\reg{d}}{\RFile(\reg{s}) \!+\! \val}
                    %R'' = R'\{ \reg{d} \rightsquigarrow \eval{\reg{s}} + \word \}
                }
            \end{minipage}
			\vspace{0.3cm}
		\end{minipage}
	}
	
	\subfigure[]
	{
		\begin{minipage}[b]{1\linewidth}
			% \small
			\scriptsize
			\begin{minipage}{0.5\linewidth}
				\[
					\infer
					{
						\itrans{(\Mem, \RFile)}{\rd \; \sr \; \reg{d}}
							{(\Mem, \RFile\{ \reg{d} \rightsquigarrow \word \})}
					}
					{
						\RFile(\sr) = \word \quad \ \ \reg{d} \in \dom(\RFile)
					}
				\]
			\end{minipage}
			\begin{minipage}{0.5\linewidth}
				\[
					\infer
					{
						\itrans{(M, R)}{\cadd \; \reg{s} \; \oexp \; \reg{d}}
						{
							(M, R\{\reg{d}
                                %  \rightsquigarrow \val_1\!+\!\val_2\})	
								\rightsquigarrow \val\})
						}
					}
					{
						R(\reg{s}) = \val_1 \quad
						{\llbracket \oexp \rrbracket}_R = \val_2 \quad
						\val = \val_1\!+\!\val_2 \quad
						\reg{d} \in \text{dom}(R)
					}
				\]
			\end{minipage}
			
			\vspace{0.2cm}
			\centering
			\begin{minipage}{1\textwidth}
				\[
					\infer
					{
						\itrans{(M, R)}{\ld \; \aexp \; \reg{d}}
						{(M, R\{\reg{d} \rightsquigarrow \val'\})}
					}
					{
						{\llbracket \aexp \rrbracket}_R = \loc \quad
						%\textbf{word\_align}(w) \quad
                        M(\loc) = \val' \quad
						\reg{d} \in \text{dom}(R)
					}
				\]
			\end{minipage}
			\vspace{0.3cm}
		\end{minipage}	
	}

	\subfigure[]
	{
		\begin{minipage}[b]{1\linewidth}
			% \small
			\scriptsize
            $$
            \begin{array}{ll}
                \begin{array}{lcl}
                    \evalR{\oexp}{R} & \define &
                    \left\{
                        \begin{array}{ll}
                            R(r) &\quad \cif \ \oexp = r \\
                            \\[-8pt]
                            w &\quad \cif \ \oexp = \word, \\
                            & \quad \quad -4096 \leq \word \leq 4095 \\
                            \\[-8pt]
                            \perp &\quad \otherwise
                        \end{array}
                    \right.
                \end{array} & \quad
                \begin{array}{lcl}
                    \evalR{\aexp}{R} & \define &
                    \left\{
                        \begin{array}{ll}
                            \evalR{\oexp}{R} &\quad \cif \ \aexp = \oexp \\
                            \\[-8pt]

                            \val_1 \!+\! \val_2
                              &\quad \cif \ \aexp = \regr \!+\! \oexp,\
                              R(\regr) \!=\! \val_1 \\
                            & \quad \quad \tand \ \evalR{\oexp}{R} = \val_2 \\

                            \\[-8pt]
                            \perp &\quad \otherwise
                        \end{array}
                    \right.
                \end{array}
            \end{array}
            $$
			\vspace{0.3cm}
		\end{minipage}	
	}
	\caption{Selected operational semantics rules. 
	(a) Program Transition. 
	(b) Control Transfer Instruction Transitions. 
	(c) Save, Restore and Wr Instruction Transitions. 
	(d) Simple Instruction Transitions. 
	(e) Expression Semantics.}
	\label{Selected Operational Semantics}
\end{figure*}
The program transition relation
$\LGlobTrans{(\code, \state, \pc, \npc)}
	{ \ \, }
	{(\code, \state', \pc', \npc')}$
is defined
in Fig.~\ref{Selected Operational Semantics} (a).
Before the execution of the instruction pointed by
$\pc$, the delayed writes
in $\DBuf$ with $0$ delay cycles are executed first.
The execution of the delayed writes is defined in the
form of $\Dstep{(\RFile, \DBuf)}{(\RFile', \DBuf')}$ below:

{\small
$$
\begin{array}{c}
\!\!
\infer
{\Dstep{(\RFile, \nil)}{(\RFile,\nil)}}
{}
\\
\\
%
% \qquad\qquad
%
\infer
{\Dstep{(\RFile, (\tick\!+\!1, \sr, \word)\dbCons\DBuf)}
       {(\RFile', (\tick, \sr, \word)\dbCons\DBuf')}}
{\Dstep{(\RFile, \DBuf)}{(\RFile', \DBuf')}}
\\
\\
\infer
{\Dstep{(\RFile, (0, \sr, \word)\dbCons\DBuf)}{(\fsubst{\RFile'}\sr\word, \DBuf')}}
{\Dstep{(\RFile, \DBuf)}{(\RFile', \DBuf')}
\qquad \sr\in\dom(\RFile)
}
%
% \qquad\qquad
\\
\\
%
\infer
{\Dstep{(\RFile, (0, \sr, \word)\dbCons\DBuf)}{(\RFile', \DBuf')}}
{\Dstep{(\RFile, \DBuf)}{(\RFile', \DBuf')}
 \qquad \sr\not\in\dom(\RFile)
}
\end{array}
$$
}

Note that the write of $\sr$ has no effect if $\sr$ is not
in the domain of $\RFile$. Since $\RFile$ is defined as a partial
map, we can prove the following lemma.
\begin{lemma}
	\label{lemma:RFileSplitExDelay}
	{\em $\Dstep{(\RFile, \DBuf)}{(\RFile', \DBuf')}$}
    and {\em $\RFile = \RFile_1 \uplus \RFile_2$},
    if and only if
    there exists {\em $\RFile_1'$} and {\em $\RFile_2'$}, such that
    {\em $\Dstep{(\RFile_1, \DBuf)}{(\RFile_1', \DBuf')}$, $\Dstep{(\RFile_2, \DBuf)}{(\RFile_2', \DBuf')}$},
    and {\em $\RFile' = \RFile_1' \uplus \RFile_2'$}.
\end{lemma}
Here the disjoint union $\RFile_1 \uplus \RFile_2$ represents the union of
$\RFile_1$ and $\RFile_2$ if they have disjoint domains, and undefined
otherwise. This lemma is important to give sound semantics
to delay buffer related assertions, as discussed in
\Sec{\ref{sec:logic}}.

%Although this would never happen
%at runtime, this feature is useful for giving sound semantics
%to delay buffer related assertions, as shown in Sec.~\ref{sec:logic}.

%We define the operation semantics of \sparc{} with multiply layers.
%We just select a part of transition rules
%in Fig. \ref{Selected Operational Semantics}
%to give readers a whole recognition for \sparc{} program execution
%because of the space limitation.
%As shown in Fig. \ref{Selected Operational Semantics},
%we define the operational semantics with four layers.
%As for the first layer Program Transition,
%we can see that
%a step of \sparc{} program transition can be split into two steps.
%The state transition ``$\Dstep{}{}$" checks each element in delay list,
%remove the delay items whose delay cycles are 0
%and write the value recorded in them to specific special register.
%Second, the instruction that \pc{} points executes.
%We define the state transition caused by delay list simply as following.
%We can find that if a special register $\sr$ recorded in $\DBuf$ isn't
%in the domain of $\RFile$. It will not make any changes for $\RFile$.

%\[
%	\small
%	\begin{array}{rcl}
%		\exedelay(R, D) & \define &
%		\left\{
%			\begin{array}{ll}
%				(R, D) & \quad  D = \nil \\
%				
%				\\[-8pt]
%				
%				(R'\{\sr \rightsquigarrow w\}, D'') & \quad D = (0, \sr, w) :: D', \\
%				& \quad \ \ (R', D'') = \exedelay(R, D') \\
%				
%				\\[-8pt]
%				
%				(R', (n-1, \sr, w) :: D'') & \quad D = (n, \sr, w) :: D', n > 0, \\
%				& \quad \ \ (R', D'') = \exedelay(R, D')
%			\end{array}
%		\right.
%	\end{array}
%\]

The transition steps for individual instructions are classified into
three categories: the control transfer steps
($\ccttrans{\notCare}{\notCare}{\notCare}$),
the steps for
$\csave$, $\crestore$ and $\cwr$ instructions
($\swtrans{\notCare}{\ \notCare\ }{\notCare}$),
and the steps
for other simple instructions
($\itrans{\notCare}{\ \notCare\ }{\notCare}$). The
corresponding step transition relations are defined inductively
in \Fig{\ref{Selected Operational Semantics}}
(b) - \ref{Selected Operational Semantics} (d)
respectively.

Note that, after the control-transfer instructions, $\pc$ is set
to $\npc$ and $\npc$ contains the target code pointer. This explains
the one cycle delay for the control transfer.
The $\call$ instruction saves $\pc$ into the register $\RAreg$,
while $\retl$ uses $\RAreg \!+\! 8$ as the
return address (which is the address for the second instruction
following the $\call$).
The conditional branch $\be{} \ \lab{}$ jumps to $\lab{}$ 
(after one-cycle delay) if 
the value in the register $\regz$ is not 0.
%justifies whether
%the branch is taken according to the value of $\regz$.}
Evaluation of
expressions $\aexp$ and $\oexp$ are defined
% $\evalR\aexp\RFile$
% and
% $\evalR\oexp\RFile$
in Fig.~\ref{Selected Operational Semantics} (e).
Here, we define the sum of two values $\val_1$
and $\val_2$ below. The result of $\val_1\!+\!\val_2$
is legal, if both of $\val_1$ and $\val_2$
are words (Int32), or $\val_1$ is an address and
$\val_2$ is an offset. The offset is a word,
which acts as an immediate value in the
calculation of address.
% This is explain why the
% immediate value in our work is a word, because
% immediate value usually acts as an offset, which
% is a word, in the calculation of address.
\[
	\small
	\val_1 + \val_2 \ \define \
	\left\{
		\begin{array}{ll}
			\word_1 + \word_2 & \quad \cif \
				\val_1 = \word_1, \text{ and }
				\val_2 = \word_2 \\
			\\[-8pt]
			(\block, \word_1 + \word_2) & \quad
				\cif \
				\val_1 = (\block, \word_1),
				\text{ and }
				\val_2 = \word_2 \\
			\\[-8pt]
			\perp & \quad \otherwise
		\end{array}
	\right.
\]

The $\cwr$ wants to save the bitwise exclusive OR of
the operands into the special register $\sr$, but
it puts the write into the delay buffer $\DBuf$
instead of updating $\RFile$ immediately.
The operation $\setdelay(\sr, \word, \DBuf)$
is defined below:
\[
	\begin{array}{l}
		\setdelay(\sr, \word, D) \define (X, \sr, \word)
		\dbCons \DBuf \\
	\end{array}
\]
where $X$ ($0 \leq X \leq 3$) is a
predefined system parameter for the delay cycle.
% Note that
% as we have explained before, special register is used
% to record the state of processors, so we do not permit saving
% memory address in it.
%whose specific value is implementation dependently.

The $\csave$ and $\crestore$ instructions rotate
the register windows and update the register file.
Their operations over $\Wstack$ and $\RFile$
are defined in Fig.~\ref{fig:save and restore}.

%Instruction $\csave$, $\crestore$ and $\cwr$
%are different with the other simple instructions
%because they may change the state of frame list and delay list.
%And we present their transition in the third layer.
%The rule for instruction $\cwr$ tells us
%that the \sparc{} set the write operation in delay list
%instead of updating the value of target special register $\sr$ immediately.
%The operation of $\setdelay(\sr, w, D)$ is defined as following,
%where $X$ ($0 \leq X \leq 3$) is the delay cycle
%whose specific value is implementation dependently :
%\[
%	\small
%	\begin{array}{l}
%		\setdelay(\sr, w, D) \define (X, \sr, w) :: D \\
%	\end{array}
%\]
%
%
%\indent
%The execution of instruction \csave{} and \crestore{} will cause the
%window rotation, so that will change the state of Frame List $F$.
%The operation $\decwin{(R, F)}$ describe that, if previous window is valid,
%we will do a right rotation for register windows by $\rightwin(R, F)$.
%As for instruction \crestore{}, it will change the register window by
%$\incwin{(R, F)}$, which will do a left rotation by $\leftwin(R, F)$ for register window.
%The auxiliary definitions for operational semantics of \csave{} and \crestore{}
%are shown in Fig. \ref{fig:save and restore}.
%
%\indent
%The last layer is designed for some simple instructions,
%whose executions only touch memory and register file.
