\section{Introduction}

Operating system kernels are at
the most foundational layer of computer software systems.
To interact directly with hardware,
many important components in OS kernels are implemented
in assembly, such as the context switch code or the code that
manages interrupts.
In addition, there is also code written directly in assembly
to achieve high performance (\eg{} \textsf{memcpy} in linux
v2.6.17.10 \cite{linuxv2.6.17.10}).
Correctness of such assembly code is crucial to ensure
the safety and security of the whole system.
However, assembly code verification remains a challenging
task in existing work on OS kernel verification
(\eg~\cite{Xu16cav, sel4, deepspec}),
where the assembly code is either unverified or verified based on
operational semantics without a general program logic.

SPARC (Scalable Processor ARChitecture)
is a CPU instruction set architecture (ISA) with high-performance
and great flexibility~\cite{sparc}.
It has been widely used in various processors
for workstations and embedded systems. %, and space mission.
The \sparc{} ISA has some interesting features, which
make it a non-trivial task to design a Hoare-style
program logic for assembly code.
%Designing a practical Hoare-style program logic
%for \sparc{} code isn't an easy work
%because we will meet many challenges as following :

\begin{itemize}
	\item \textit{Delayed control transfers}.
	\sparc{} has two program counters \pc{} and \npc.
    The \npc{} register points to the next instruction
    to run.
	Control-transfer instructions in \sparc{}
	change \npc{} instead of \pc{} to the target program point,
    while \pc{} takes the original value of \npc.
	This makes the control transfer to happen one cycle later
    than the execution of the control transfer instructions.
	
	\item \textit{Delayed writes}.
	The $\cwr$ instruction that writes a special class
    of registers such as the window invalid mask register
    \regwim{} does not take effect immediately.
    % $\regwim$ used to justify
    % whether a register window is valid does not take
    % effect immediately.
    That is, ``they may take until
    completion of the third instruction following the write
    instruction to consummate their write operation.
    The number of delay instructions (0 to 3)
    is implementation-dependent"~\cite{sparc}.

	% The $\cwr$ instruction that writes a special class
    % of registers
    % does not take effect immediately.
    % % in the current cycle when
    % %pointed to \pc.
    % Instead the write operation is buffered
    % and then executed $X$ cycles later, where $X$ is a
    % predefined
    % system parameter which usually ranges from 0 to 3.

	\item \textit{Register windows}.
	\sparc{} uses register windows
	and a window rotation mechanism
	to avoid saving contexts in the stack directly
    and achieves high performance in context management.
    % {\color{blue}
    % As the example shown in
    % \Fig{\ref{fig:An Example for SPARC Code}},
    % the \texttt{CALLER} doesn't save its context in
    % stack in memory when calling function
    % \texttt{ChangeY},
    % but use an instruction \csave{} to save
    % context by window rotation.
    % }
	%and a significant reduction in context management.
\end{itemize}

We use a simple example
in Fig.~\ref{fig:An Example for SPARC Code}
to show these three features.
\vspace*{-1em}
\begin{center}
    % \subfigure[]
    % {
    %     \begin{minipage}[b]{0.45\textwidth}
    %         \[
    %             \begin{array}{ll}
    %                 \multicolumn{2}{l}
    %                 {\texttt{CALLER}:} \\[1ex]
    %                 \quad & \vdots \\
    %                 \text{1} \ \ & \mov \quad 1, \; \oreg{0} \\
    %                 \text{2} & \call \quad \text{ChangeY} \\
    %                 \text{3} &
    %                     \csave \quad \spreg, \; -64, \; \spreg \\
    %                 \text{4} &
    %                     \mov \quad \oreg{0}, \; \lreg{0} \\
    %                 \quad & \vdots
    %             \end{array}
    %         \]
    %     \end{minipage}
    % }
    % \subfigure[]
    % {
    %     \begin{minipage}[b]{0.45\textwidth}
    %         \[
    %             \begin{array}{ll}
    %                 \multicolumn{2}{l}
    %                 {\texttt{ChangeY}:} \\[1ex]
    %                 \text{5} \ \ &
    %                 \rd \quad \regY, \; \lreg{0} \\
    %                 \text{6} &
    %                     \cwr \quad \ireg{0}, \; 0, \; \regY \\
    %                 \text{7} & \nop \\
    %                 \text{8} & \nop \\
    %                 \text{9} & \nop \\
    %                 \text{10} & \ret \\
    %                 \text{11} &
    %                 \crestore \quad \lreg{0}, 0, \oreg{0}
    %             \end{array}
    %         \]
    %     \end{minipage}
    % }
    $
        \begin{array}{ll}
            \texttt{CALLER}: &
            \texttt{ChangeY} : \\[1ex]
            \begin{array}{ll}
                \quad & \vdots \\
                \text{1} \ \ & \mov \quad 1, \; \oreg{0} \\
                \text{2} & \call \quad \text{ChangeY} \\
                \text{3} &
                    \csave \quad \spreg, \; -64, \; \spreg \\
                \text{4} &
                    \mov \quad \oreg{0}, \; \lreg{0} \\
                \quad & \vdots 
            \end{array} & \
            \begin{array}{ll}
                \text{5} \ \ &
                    \rd \quad \regY, \; \lreg{0} \\
                \text{6} &
                    \cwr \quad \ireg{0}, \; 0, \; \regY \\
                \text{7} & \nop \\
                \text{8} & \nop \\
                \text{9} & \nop \\
                \text{10} & \ret \\
                \text{11} &
                    \crestore \quad \lreg{0}, 0, \oreg{0}
            \end{array} \\ \\[-9pt]
            \qquad\quad \text{(a)} & 
            \qquad\quad \text{(b)}
        \end{array}
    $
    \figurecaption{An Example for SPARC Code. 
    (a) The function \texttt{CALLER}. (b) The function \texttt{ChangeY}.}
    \label{fig:An Example for SPARC Code}
\end{center}

\sparc{} has 32 general registers, which are split into four
logic groups
as \textsf{global} ($\reg{0} \sim \reg{7}$),
\outRN{} ($\reg{8} \sim \reg{15}$),
\localRN{} ($\reg{16} \sim \reg{23}$) and
\inRN{} ($\reg{24} \sim \reg{31}$) registers.
Correspondingly, we give aliases ``$\greg{0} \sim \greg{7}$'',
``$\oreg{0} \sim \oreg{7}$'', ``$\lreg{0} \sim \lreg{7}$''
and ``$\ireg{0} \sim \ireg{7}$'' for these groups
respectively.

\texttt{CALLER} in Fig.~\ref{fig:An Example for SPARC Code} (a)
calls \texttt{ChangeY},
which updates the {\em special register} \texttt{Y}
and returns its original value.

% \begin{figure*}[!t]
% 	\vspace{-0.5em}
% 	\[
% 		\begin{array}{ll}
%             \texttt{CALLER}: &
%             \quad\hspace*{2.5ex} \texttt{ChangeY} : \\[1ex]
% 			\begin{array}{ll}
% 				\quad & \dots \\
% 				\text{1} \quad & \mov \quad 1, \; \oreg{0} \\
% 				\text{2} \quad & \call \quad \text{ChangeY} \\
%                 \text{3} \quad &
%                     \csave \quad \spreg, \; -64, \; \spreg \\
%                 \text{4} \quad &
%                     \mov \quad \oreg{0}, \; \lreg{0} \\
% 				\quad & \dots \\
% 				\quad
% 			\end{array} & \quad\quad
% 			\begin{array}{ll}
%                 \text{5} \quad &
%                     \rd \quad \regY, \; \lreg{0} \\
%                 \text{6} \quad &
%                     \cwr \quad \ireg{0}, \; 0, \; \regY \\
% 				\text{7} \quad & \nop \\
% 				\text{8} \quad & \nop \\
% 				\text{9} \quad & \nop \\
% 				\text{10} \quad & \ret \\
%                 \text{11} \quad &
%                     \crestore \quad \lreg{0}, 0, \oreg{0}
% 			\end{array}
% 		\end{array}
% 	\]
% 	\caption{An Example for SPARC Code}
% 	\label{fig:An Example for SPARC Code}
% 	\vspace{-0.5em}
% \end{figure*}

\texttt{ChangeY} in \Fig{\ref{fig:An Example for SPARC Code}} (b)
requires an input parameter
as the new value for the special register {\tt Y}.
\texttt{CALLER} calls \texttt{ChangeY} at line 2,
and \pc{} and \npc{} point to lines 2 and 3 respectively at this
moment.
The call instruction changes the value of \pc{} to \npc{}
and lets \npc{} point to the entry of
\texttt{ChangeY} at line 5,
which means the control-flow will not transfer to \texttt{ChangeY}
in the next cycle,
but in the cycle after the execution of the $\csave$
instruction following the call. Similarly, when
\texttt{ChangeY} returns (at line 10), the control is transferred
back to the caller after executing the $\crestore$
instruction at line 11.
We call this feature ``delayed control transfers''.

%\sparc{} program won't save current context when occurring function call.
\sparc{} uses the $\csave$ instruction (at line 3 in the example)
to save the current context
and $\crestore$ instruction (at line 10) to restore it.
As explained above, among the 32 general registers,
the \outRN{}, \localRN{} and \inRN{} registers form the
{\em current register window}.
The \localRN{} registers are for private use in the current context.
The \inRN{} and \outRN{} registers are shared with adjacent register windows
for parameters passing.
The $\csave$ instruction rotates the register window from the
current one to the next. Then the \localRN{} and \inRN{}
registers in the original window are no longer accessible,
and the original \outRN{} registers become the \inRN{} registers
in the current window.
In \Fig{\ref{fig:An Example for SPARC Code}},
the \texttt{CALLER} uses the \csave{} instruction 
(at line 3) to save its context
(\localRN{} and \inRN{} registers) and rotate the
register window (so that the 
\outRN{} registers become the \inRN{} registers).
So, the $\ireg{0}$ register
assessed in \texttt{ChangeY} at line 6 is
the same register as the $\oreg{0}$ modified in
\texttt{CALLER} at line 1.
The $\crestore$ instruction does the inverse.
The arguments taken by the $\csave$ and  $\crestore$ instructions
are irrelevant here and can be ignored.
% Note that the arguments taken by the $\csave{}$
% in this example says,
% in addition to operate the register window,
% the execution of this instruction will also
% allocate a new stack frame size 64 bytes
% to \texttt{ChangY}. Here, $\spreg$ register,
% which is an alias of $\reg{14}$, usually
% acts as the stack pointer in SPARCv8.
% The $\crestore{}$ instruction does the inverse
% and the arguments it takes are irrelevant here
% and can be ignored.

% assign the value of ($\spreg\!-\!64$), where the value
% of the register $\spreg$ is taken
% from the original window, to the $\spreg$ register
% in the new register window.
% The $\spreg$ register, which is an alias of $\reg{14}$,
% acts as the stack pointer.
% So, `` $\csave{} \ \spreg, -64, \spreg$ "
% in this example allocates a new stack frame size 64 bytes
% to the callee (\texttt{ChangY}).
% % The arguments taken by the $\crestore{}$
% % are irrelevant here and can be ignored here.
% Because the $\crestore{}$ instruction restores
% the stack pointer of the caller, the stack frame
% allocated for callee will be freed after executing
% $\crestore{}$.

At line 6, the $\cwr$ instruction tries to
update the special register {\tt Y}
with the value of $\ireg{0} \! \xorOP 0$
(bitwise exclusive OR).
Note that the $\ireg{0}$
register here is the same register as $\oreg{0}$ at line 1.
However, the write is delayed for $X$ cycles,
where $X$ is some predefined system parameter
that ranges from 0 to 3.
For portability, programmers usually do not rely
on the exact value of $X$ and assume it takes
the maximum value 3.
Therefore three \nop{} instructions
are inserted.
Reading of {\tt Y} earlier than line 9
may give us the old value.
This feature is called ``delayed writes''.

These features make the semantics of the \sparc{} code
context-dependent. For instance, a read of a special register
(\eg{} the register {\tt Y} in the above example) needs to
make sure there are enough instructions executed since the
most recent {\em delayed} write. As another example,
the instruction following the \call{} can be any
instruction in general, but it is not supposed to
update the register $\RAreg$, which contains the return
address saved by the \call{} instruction.
In addition,
the delayed control transfer
and the register windows also allow highly flexible calling
conventions. Together, they make it a challenging task
to have a Hoare-style program logic for local and modular
reasoning of \sparc{} assembly code.

Working towards a certified OS kernel for aerospace
crafts whose inline assembly is written in \sparc,
we try to address these challenges and propose a practical
program logic for realistically modelled \sparc{} code.
We have applied our logic to verify the main body of
the task context switch routine in the kernel~\cite{zha18aplas}.
% \mycomments{Xinyu: Cite the APLAS paper here}

However, the OS kernel is implemented as
C language mainly and SPARCv8 as inline assembly.
Just having a traditional Hoare-style
program logic for SPARCv8, which can only make sure
the safe execution of SPARCv8 program if the initial
state satisfies the precondition, is insufficient.
\etal{Xu}~\cite{Xu16cav} proposed a program logic for
verifying the correctness of OS kernel implemented in
C language with inline assembly,
but they used {\em abstract assembly primitives} to
substitute the inline assembly in their verification work.
As a supplement to their work, we extend our
program logic so that it can ensure the
contextual refinement relation,
shown in (\ref{eq:refinement}),
between the implementations and
their corresponding abstract assembly primitives.
Here, we use $\Clang$, $\AbsPrimC$,
and $\asmimp$ to represent the C language program,
the set of abstract assembly
primitives and the implementations of
abstract assembly primitives respectively.
It means $\asmimp$ refines $\AbsPrimC$ under
\textit{any context} $\Clang$.
\begin{equation}
    \forall \, \Clang. \,
    \Clang[\asmimp] \refine \Clang[\AbsPrimC]
    \label{eq:refinement}
\end{equation}
%
Here we use ``$\refine$'' to represent the refinement
relation.
%The $\Clang[\asmimp] \refine \Clang[\AbsPrimC]$ represents
%that $\Clang[\asmimp]$ refines $\Clang[\AbsPrimC]$.}
%

However, if we use C program as a client code to
call inline assembly code, we need to define the
semantics of C-assembly interaction.
%It may be possible to use
%\textit{interaction semantics}~\cite{Stewart15popl}
%to implement multi-language linking, but
%there is a contraint in interaction semantics
%requiring that a callee must always return to its caller when
%finished, and therefore it can't handle calling the context switch
%routine, which uses the return address on a different stack and jumps
%to a different caller.
\begin{center}
    $
        \begin{array}{l}
            \semantics{\Clang[\AbsPrimC]}^{\textsf{C}}
            \\
            \\[-5pt]
            \!\!
            {\color{blue} (\text{S1})} \ \
            \rotatebox[origin=c]{90}{$\refine$} \ \
            \Longleftarrow \ \
            \boxed{C = \textit{Comp}(\Clang)}
            \\
            \\[-5pt]
            \semantics{\code[\hprimset]}^{\textsf{P-SPARCv8}}
            \\
            \\[-5pt]
            \!\!
            {\color{blue} (\text{S2})} \ \
            \rotatebox[origin=c]{90}{$\refine$} \ \
            \Longleftarrow \ \
            \boxed{\vdash\asmimp:\hprimset}
            \\
            \\[-5pt]
            \semantics{\code[\asmimp]}^{\textsf{SPARCv8}}
        \end{array}
    $
    \figurecaption{Idea to establish contextual refinement}
    \label{fig:idea to establish contextual refinement}
\end{center}

Since the goal of this paper is to verify the correctness of the
assembly code with respect to the abstract assembly primitives,
we want to avoid the C-assembly interaction (and leave it as
future work). We decompose the OS verification tasks into
two steps, as shown in \Fig{\ref{fig:idea to establish contextual refinement}},
and focus on {\color{blue} (S2)} only in this paper.
% some constraints in interaction semantics and
% features of SPARCv8 make it difficult to use
% interaction semantics in our work.
% \textbf{First}, interaction semantics requires that
% the callee must return to caller when finished.
% So, it can't handle calling context switch routine.
% \textbf{Second}, it's also difficult to instantiate
% the SPARCv8 language in interaction semantics.
% The features of SPARCv8 makes the semantics of
% SPARCv8 code context-dependent. For example,
% the caller of

%We take the approach shown in
%\Fig{\ref{fig:idea to establish contextual refinement}}.

%We consider a method to establish the contextual
%refinement between implementations and their
%corresponding abstract assembly primitives without
%cross-language interaction.
%We use \Fig{\ref{fig:idea to establish contextual refinement}}
%to illustrate our idea.

% We introduce this figure from top to down.
The source program of OS kernel
$\Clang[\AbsPrimC]$, which executes
under the C language semantics (shown as
$\semantics{\notCare}^{\textsf{C}}$),
is implemented as C language
with a set of abstract assembly primitive $\AbsPrimC$.
The compiler (\textit{Comp}) translates the
C program $\Clang$ to \sparc{} code $\code$.
As {\color{blue} (S1)} shows,
we assume the compilation ensures the
refinement between $\Clang[\AbsPrimC]$ and
$\code[\hprimset]$ that
executes under Pseudo-SPARCv8 semantics shown as
$\semantics{\notCare}^{\textsf{P-SPARCv8}}$.
Here, $\hprimset$ represents the set of
abstract assembly primitives in the middle layer.
We use distinguished notations to represent
the set of abstract assembly primitives in source
and intermediate level, because they execute on
different program states and have different semantics.
The Pseudo-SPARCv8 language $\code[\hprimset]$,
which uses \sparc{} as client code and is able to
call abstract assembly primitives in $\hprimset$.
In {\color{blue} (S2)},
we verify using our refinement logic that
the whole \sparc{} program
$\code[\asmimp]$ executing under the realistically
modelled SPARCv8 semantics, represented as
$\semantics{\notCare}^{\textsf{SPARCv8}}$, refines
the program $\code[\hprimset]$ executing
under the Pseudo-SPARCv8 semantics.
Finally, we can get
$\semantics{\code[\asmimp]}^{\textsf{SPARCv8}}
\refine
\semantics{\Clang[\AbsPrimC]}^{\textsf{C}}$.
In this work, we focus on
{\color{blue} (S2)}, and
leave {\color{blue} (S1)}
as future work.

% Then, considering some works, \eg{\cite{Xu16cav, Feng08pldi}},
% define a set of abstract assembly primitives,
% which are atomic operations and act as the specifications of
% the assembly functions,
% to substitute their concrete implementations
% for some convenience of program verification.
% % Then, considering some works, \eg{\cite{Xu16cav, Feng08pldi}},
% % define an abstract assembly
% % primitive $\primsw{}$ to substitute the concrete
% % implementation of context switch rontine,
% We extend our program logic to support refinement
% verification to bridge the gap.
% The extended program logic ensures the
% {\it contextual refinement} relation \cite{liang13pldi,Xu16cav}
% between the implementation
% of SPARCv8 function and its corresponding abstract assembly
% primitive. The {\it contextual refinement} can be represented
% as the following form, which means that
% calling a SPARCv8 function $\asmimp$
% will not produce more observable behaviors than calling
% its corresponding abstract assembly primitive $\hprim$
% under {\it any context} $\code$:
% % the abstract assembly primitive $\hprim$ will not produce more
% % observable behaviors than calling their implementation $\asmimp$
% % under {\it any context} $\code$:
% \[
%     \forall \ \code. \, \code[\asmimp] \subseteq
%         \code[\hprim]
% \]

Our work is based on earlier work on assembly code
verification but makes the following contributions.
% \footnote{We list six contributions here. The
% first four contributions have already presented
% in the preliminary version~\cite{zha18aplas}
% of this paper. And the
% additional two contributions about refinement
% verification are our new contributions.}:

%The logic, its soundness
%proof, and the verification of the context switch module
%are all mechanized in the Coq proof assistant.

% we extend our program logic to support refinement
% verification. We formuate and verify the correctness of
% inline assembly SPARCv8 functions as {\it contextual refinement}
% their implementation and its corresponding abstract assembly
% primitive, like context switch primitive \textbf{switch} defined
% in \cite{Xu16cav}. Our extended program logic implies this
% {\it contextual refinement} relation, and can make sure that the program
% calling the abstract assembly primitives will not produce more
% observable behaviors than their implementations.
% Our work is based on earlier work on assembly code
% verification but makes the following contributions:

%\indent
%In this paper,
%we propose a novel
%and practical Hoare-style program logic
%for a realistically modelled machine code, \sparc{}.
%We not only solve the challenges above,
%but also show its application
%in verifying a key module in OS kernel, context switch.
%To summarize, our work makes the following contributions:

\begin{itemize}
    \item
    We propose a new program logic which supports relational reasoning for
    refinement verification.
    It ensures that a verified \sparc{} function
    contextually refines its corresponding
    abstract assembly primitive.
%
    Our logic supports all the above features of \sparc.
    We redefine basic blocks to include the instruction
    following the jump or return as the tail of a
    block, which models the delayed control
    transfer. To reason about delayed writes, we introduce
    a modal assertion $\dlyregst{t}\sr\word$, saying that
    the special register $\sr$ will
    hold the value $\word$ in up to $t$ cycles.
    %$\dlyast{t}\astP$, saying that $\astP$
    %becomes true in up to $t$ cycles. In particular,
    %$\dlyregst{t}\sr\word$ means the special register $\sr$ will
    %hold the value $\word$ in up to $t$ cycles.
    We also give
    logic rules for $\csave$ and $\crestore$ instructions
    that do register window rotation.
%
%
%	We present a program logic
%	that has been designed to fit on
%	accurate models of \sparc{} language,
%	so that we can verify the \sparc{} program modularly.
%	We does not only care about the
%	safety property of the \sparc{} codes in our work,
%	but also their functional behavior.
%	Our logic can support main features of \sparc{} code,
%	including delayed control-transfer,
%	register windows, and delayed-write.
    \item
    In order to support refinement verification, we define a
    Pseudo-SPARCv8 language as the language to implement
    the high-level specification.
    % which
    % is multi-threaded and can call abstract
    % assembly primitives.
    It also hides the details of
    sophisticated register window mechanism
    in \sparc{} by abstraction,
    and makes its language model
    simpler than realistic \sparc{}.
    % and makes the program state
    % simpler than physical \sparc{} program
    % (defined in \Sec{\ref{sec:modeling}}).
    So, it can provide some convenience
    to write the abstract assembly primitive
    and reason in Pseudo-SPARCv8 level.

	
	\item
    Following SCAP~\cite{Feng06pldi}, our logic supports
    modular reasoning of function calls in a direct-style.
    However,
    we use the traditional pre- and post-conditions as function
    specifications, instead of the assertion $g$ used
    in SCAP. This allows us to reuse existing techniques
    (\eg{} Coq tactics) to simplify the %program
    verification process.
    The logic rules for function call and return are general
    and independent of any specific calling convention.

    \item
    We give direct-style semantic interpretation for
    the logic judgments, based on which we establish the
    soundness. This is different from previous
    work, which either does syntactic-based soundness proof
    (\eg{} SCAP~\cite{Feng06pldi}) or treats return code pointers
    as first-class code pointers and gives CPS-style
    (continuation-passing style) semantics.
    Those approaches for soundness make it difficult to verify
    the interaction between the inline assembly and the C
    code in the kernel, the latter being verified following
    a direct-style program logic.

    \item
	Context switch of concurrent tasks is an important
    component in OS kernels. It is usually implemented
    as inline assembly because of the need to access
    registers and the stack. We apply our logic and
    verify that there is a contextual refinement
    between a context switch routine implemented
    in \sparc{} and an abstract switch primitive.

	% We verify the main body of the context switch routine
	% in a realistic embedded OS kernel for aerospace crafts,
    % which consists of around 250 lines of \sparc{} code.

\end{itemize}
The program logic and its soundness proof
have been mechanized in Coq. Coq proofs and a compansion 
technical report are available at 
\url{}
% \footnote{More details about our work are
% available at TR and the Coq code.
% \url{https://sites.google.com/site/jcstarticle1/}.}.
% including the extended program logic
% for refinement verification, and its soundness proof
% have been mechanized in Coq
% \footnote{The implememntation of the program Logic for
% \sparc{} in {Coq} (project code).
% \url{https://github.com/jpzha/VeriSparc}.}.

This paper extends the conference paper in
APLAS 2018~\cite{zha18aplas}.
The program logic there can only verify the partial
correctness of \sparc{} code and we make the following
expansions.
\begin{itemize}
    \item We propose a new program logic to do
        relational reasoning for refinement verification
        (\Subsec{\ref{subsec:rellogic}} for details).
    \item In order to support refinement verification,
        this paper presents a new Pseudo-SPARCv8 language
        as the high-level
        specification language
        (\Subsec{\ref{subsec:High-level Pseudo-SPARCv8 Language}}
        for details).
    \item We also use the new logic to verify the implementation
        of a context switch routine, by showing that
        the implementation contextually
        refines an abstract switch primitive
        (\Sec{\ref{sec:ctxswitch}} for details).
\end{itemize}

In the remainder of the paper,
%we first introduce the background of our work
%and give an overview of our approach.
we present the program model
and operational semantics of \sparc{} in 
\Sec{\ref{sec:modeling}}.
For clear presentation, we use a
simplified version of our program logic that
does not support refinement verification to
demonstrate how our logic supports the three main
features of \sparc{} in \Sec{\ref{sec:logic}},
which is the main point
of our work and irrelevant with
refinement verification.
We present the Pseudo-SPARCv8 program and
the relational program logic for refinement reasoning
% how
% our logic supports refinement verification
in \Sec{\ref{sec:refine-verification-sparc}}.
We show the verification of a context switch routine
in \sparc{} in \Sec{\ref{sec:ctxswitch}}.
% the main body of the context switch routine
% in Sec.~\ref{sec:ctxswitch}.
% The extended program logic to
% support refinement verification is presented in
% \Sec{\ref{sec:refine-verification-sparc}}, including
% the high-level Pseudo-SPARCv8 program introduced in
% \Sec{\ref{subsec:High-level Pseudo-SPARCv8 Language}}.
% >>>>>>> New
Finally, we discuss more on
related work in \Sec{\ref{sec:related work}}
and conclude in 
\Sec{\ref{sec:conclusion}}.
%Sec.~\ref{sec:ctxswitch} gives an actual application of our logic
%to verify a realistic context switch module in an embedded OS kernel.
%Finally, we discuss more related work
%and conclusion in Sec. \ref{sec:conclusion}.
